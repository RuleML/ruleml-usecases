%%-------------Abstract-----------------
\chapter*{Abstract}
\addcontentsline{toc}{chapter}{Abstract} \doublespacing 

Tourism is the world's largest and fastest growing industry. There are conventional tourism service providers which are competitively trying to provide the best travel services to customers based on their interests. The Semantic Web is a major endeavour to enhance the Web by enriching its content with semantic (meta)data that can be processed by inference-enabled Web applications. eTourism is a good candidate for such enrichment, since it is an information-based business. As with any such business, providing the relevant information for the consumer means a better end product. Thus, providing a well-structured and comprehensive Knowledge Base (KB) for consulting will bolster the eTourism business.
In this thesis, we have designed and implemented a KB consisting of tourism domain-specific information. Our KB stores facts about Bhutan, which are structured by a light-weight ontology (adapted from the Harmonise eTourism ontology) and used by partonomy rules that encode the geographical partitioning of regions and provide a basis for activity search capabilities. On top of these, planning rules are applied to deduce recommendations of routes, activities (attractions and events), and accommodations. This thesis also discusses transferring Friend Of A Friend (FOAF) concepts for semantically describing persons or organizations, to tourist-entity profiles. The FOAF-like Harmonise relation ``relatedTo" between tourist entities is used to chain through provinces and attraction profiles, hence to provide attraction-centric recommendations. This prototype, eTourPlan, an eTourism planner using Semantic Web techniques has been implemented in RuleML/POSL. Results of running eTourPlan in the prototype RuleML engine OO jDREW are reported.


