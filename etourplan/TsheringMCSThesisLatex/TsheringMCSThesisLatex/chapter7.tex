%%-----------Chapter 7------------------------------------------
\chapter{Conclusion}
\hspace{0.3in} Our eTourPlan prototype is a knowledge-based tourist route and activity planner.  The KB is comprised of object-centric facts of tourist entities, which are structured by light-weight ontologies of the tourism subdomains.  This well-structured and comprehensive KB is complemented with rule subsystems needed for providing various tourist services such as precise search, tour recommendations,  and  travel plans.  In order to show a real-world implementation of this prototype, we have developed and evaluated a KB based on the tourism information of Bhutan.

\section{Contributions}
\hspace{0.3in} This thesis was mainly focused to design, implement, and evaluate an eTourism prototype for Bhutan. We have designed light-weight ontologies (adapted from the Harmonise eTourism ontology) using RDFS to capture the tourism subdomains. We then built a fact base based on Bhutan tourist information for describing tourist entities using FOAF-like profile facts.  The ontology-structured fact base in our KB is enriched with various rule subsystems needed for generating a travel plan. Query rules were used to perform semantic searches. Partonomy rules were implemented for the administrative subdivision of a country. Derivation rules were used to deduce transitive closure facts used for route and distance computation. Finally, inference rules were implemented for  providing planning and recommendation modes of services. The eTourPlan's three distinct operations: search, recommendation, and planning are evaluated with varying user preferences in the OO jDREW reasoning engine.


\hspace{0.3in}Our eTourPlan prototype has explored the reasoning potentials of rules on a complete KB of a multi-dimensioned domain (i.e. the tourism domain). Our KB consists of 115 classes, 73 facts about Bhutan tourist information, and 37 rules in total, which cover Bhutan's entire administrative partonomy, route computation for the entire road-map between provinces, profiles of 10 (of 20) provinces and tourist information for these. Results of running the eTourPlan rule system in the rule engine OO jDREW show that our knowledged-based eTourPlan prototype can offers multiple options for a diversity of travel plans with recommendations. Each of the resulting travel plan provides a detailed information of selected events, recommended attractions and routes between event locations. Evaluation of the prototype also shows that it can provide precise parametric search results for queries on the tourism KB.

\section{Future Work}
\hspace{0.3in}The eTourPlan prototype currently only supports complete planning where users provide a fixed number of preferences and the planner schedules the entire trip for the user. We have discussed other planning strategies such as partial planning and sequence planning in this thesis. A good example of partial planning is where users specify some individual events or attractions of their choice and ask for N activities in total. An example of sequence planning is where user provides a list of activites and wants the system to order their trip. Adding such sophisticated planning options would provide a very flexible service to our user. 

\hspace{0.3in}In this thesis, we have considered the cost measure only for the accommodation subdomain. An additional feature of cost estimation for the total travel will complement the current planner. Similar approaches used for validation of total number of days for the travel can be implemented.

\hspace{0.3in}Our eTourPlan prototype is an executable ``specification" of a complete travel planning using Semantic Web techniques. Although, it was tested successfully on the OO jDREW TD reasoning engine, the execution times of the top-level predicates are relatively slow mainly because OO jDREW TD performs a complete search of all subpredicates for each iteration of recursive predicates. The OO jDREW team is currently working towards upgrading the reasoning engine to a more efficient  indexing system. Therefore, when this new version of OO jDREW is complete, a complete re-evaluation of the eTourPlan on this improved reasoning engine is highly recommended.

\hspace{0.3in}This executable specification of a knowledge-based tourist route and activity planner implemented for this thesis can also be integrated with a database or translated to a self-contained database application.

\hspace{0.3in}The facts in the KB used in this thesis are handcrafted. These facts could be either derived from some databases or extracted from RDFWeb pages. Currently, RDFWeb is used as a means to describe machine-understandable person-centric profiles \cite{DB:03}. This approach can be used to semantically describe a Web of inter-related homepages of tourist entities using the W3C's  RDF technology to integrate information from different homepages and connect them with respect to different relationships. 

\hspace{0.3in}Lastly, designing a user-friendly graphical user interface would increase the utility of the key operations of the
eTourPlan prototype. The semantic model and search of eTourPlan can be extended to a
�Semantic Bhutan" portal (and transferred to other regions such as New
Brunswick).